%% LyX 2.2.3 created this file.  For more info, see http://www.lyx.org/.
%% Do not edit unless you really know what you are doing.
\documentclass[english]{article}
\usepackage[LGR,T1]{fontenc}
\usepackage[latin9]{inputenc}
\usepackage{amsmath}
\usepackage{amssymb}

\makeatletter

%%%%%%%%%%%%%%%%%%%%%%%%%%%%%% LyX specific LaTeX commands.
\DeclareRobustCommand{\greektext}{%
  \fontencoding{LGR}\selectfont\def\encodingdefault{LGR}}
\DeclareRobustCommand{\textgreek}[1]{\leavevmode{\greektext #1}}
\ProvideTextCommand{\~}{LGR}[1]{\char126#1}


%%%%%%%%%%%%%%%%%%%%%%%%%%%%%% User specified LaTeX commands.
\renewcommand{\vec}[1]{\mathbf{#1}} 

\makeatother

\usepackage{babel}
\begin{document}

\title{Extracting mutational signatures using LASSO}

\author{Avantika Lal, Daniele Ramazzotti}
\maketitle

\section{Mutational signatures}

Point mutations occurring in a genome can be divided into 96 categories
based on the base being mutated, the base it is mutated into and its
two flanking bases. Therefore, for any patient, it is possible to
represent all the point mutations occurring in that patient's tumor
as a vector $\mathbf{m}$ of length 96; $\mathbf{m}=$ {[}$m_{1...}$$m_{96}${]}
, where each element $m_{j}$ represents the count of mutations of
category $j$ in the patient. 

A mutational signature represents the pattern of mutations produced
by a mutagen or mutagenic process inside the cell. Each signature
can be represented by a vector $\mathbf{s}$ of length 96; $\mathbf{s}=$
{[}$s_{1...}$$s_{96}${]} where each element $s_{j}$ represents
the probability that this particular mutagenic process generates a
mutation of category $j$. Since these are probabilities, $\sum\limits _{j=1}^{96}s_{j}=$
1.

A patient's tumor genome can be exposed to multiple mutagenic processes,
at different intensities. Therefore the vector of mutations in a single
patient's tumor can be considered to be a weighted sum of K mutational
signatures (K is an unknown number).

\begin{equation}
\mathbf{m}=\sum\limits _{i=1}^{K}\alpha_{i}\mathbf{s_{i}}
\end{equation}

Where $\alpha_{i}$ is the exposure of the patient to the mutagenic
process with signature $\mathbf{s_{i}}$. 

Our goal here is to extract the mutational signatures that best explain
the mutation counts of a large number of patients.

\section{Matrix Representation}

When dealing with multiple patients, we can represent their mutation
counts in matrix form. $M_{n\times J}$ is the matrix of counts, where
each row represents the $i^{th}$ patient and each column represents
the $j^{th}$ category. $M_{ij}$ is the number of mutations of the
$j^{th}$ category in the $i^{th}$ patient.

n = Number of patients

J = number of mutation categories (96)

K = number of signatures 

Alexandrov et al. have represented $M$ as follows:

\begin{equation}
M=\alpha\beta
\end{equation}

Where

$\alpha_{n\times K}$ is the weights matrix. $\alpha_{ij}$ is the
weight for the $j^{th}$ signature in the $i^{th}$ patient.

$\beta_{K\times J}$ is the signature matrix, where each row represents
a signature. $\beta_{ij}$ is the proportion of mutations in the $i^{th}$
signature that fall into the $j^{th}$ category.

They then use NNMF (non-negative matrix factorization) to solve equation (2) for $\alpha$ and $\beta$.

\section{Improvements to the method of Alexandrov et al.}

We propose two improvements to this method:
\begin{enumerate}
\item We introduce a null model based on genome frequencies of trinucleotides.
This represents the pattern of mutations that would be expected by
random chance, without any specific mutational process. Effectively,
it is a signature of randomness. We represent this by a vector $\mathbf{m_{0}}$
of length J.
\item We also introduce a sparsity constraint. While signatures are J-dimensional
vectors, each mutagenic process in the cell is expected to affect
only a few specific trinucleotides. Therefore, we expect the matrix
\textgreek{b} to be sparse.
\end{enumerate}
Based on these, we now represent the matrix $M$ as:

\begin{equation}
M=\mathbf{\alpha_{0}^{\intercal}}\mathbf{m_{0}}+\alpha\beta+\lambda\lVert\beta\lVert
\end{equation}

Where $\mathbf{\alpha_{0}}$ is a vector of weights of length n.

\section{Solving}

Given $M,$ $\mathbf{m_{0}}$ and $\lambda$, we can obtain the maximum
likelihood values of $\alpha$ and $\beta$ using a three-step approach. The first two steps are performed in an iterative EM (expectation maximization) manner to maximize the fit and enhancing sparsity in the resulting signatures. 
\begin{enumerate}
\item Fit $M=\mathbf{\mathbf{\alpha_{0}^{\intercal}}}\mathbf{m_{0}}+\alpha\beta$ using
nnls (non-negative least squares)
\item Fit the residual $M - \mathbf{\mathbf{\alpha_{0}^{\intercal}}}\mathbf{m_{0}} = \alpha\beta+\lambda\lVert\beta\lVert$ using nnlasso (Non-Negative Lasso).
Currently we use the signatures derived from NMF as a starting point for our first two steps.
\item At the end of the EM, normalize each row of $\beta$ to sum to 1 to get rates.
\end{enumerate}
Here we assume that the null model $\mathbf{m_{0}}$ is known. If
it is unknown, it is also possible to infer both $\mathbf{m_{0}}$
and $\mathbf{\alpha_{0}}$ using SVD.

\section{Cross-validation to choose $\mathbf{\lambda}$ and K}

We can use the following cross-validation strategy to choose optimal
values of $\lambda$ and K:
\begin{enumerate}
\item Randomly leave out 10\% of the cells in the matrix $M$ to obtain
matrix $M'$. 
\[
M'=\begin{pmatrix}m_{11} & [] & m_{13}\cdots & m_{1J}\\{}
[] & m_{22} & m_{23}\cdots & m_{2J}\\
\vdots & \vdots & \ddots & \vdots\\
m_{n1} & \cdots & [] & m_{nJ}
\end{pmatrix}
\]
\item Solve equation (3) using $M'$ in place of $M$ (We have to somehow
impute the missing values).
\item Predict the missing values.
\item Compute average prediction error for multiple values of $\lambda$
and K.
\item Select values of $\lambda$ and K that minimize prediction error.
\end{enumerate}

\section{Accounting for individual copy number variation}

Tumors have copy number changes, so the actual trinucleotide frequency
is expected to differ slightly from patient to patient. Thus there
should ideally be n indipendent null models (one for each patient). If we wish
to account for this, we can define the null model $M_{0}$ as a n
x J matrix.

\begin{equation}
M=\mathbf{\alpha_{0}^{\intercal}}\cdot M0+\alpha\beta+\lambda\lVert\beta\lVert
\end{equation}

Where:

\[
\mathbf{\alpha_{0}^{\intercal}}\circ M0=(\alpha_{i}\cdot m_{ij})=\begin{pmatrix}\alpha_{1}\cdot m_{11} & \cdots & \alpha_{1}\cdot m_{1J}\\
\vdots & \ddots & \vdots\\
\alpha_{n}\cdot m_{n1} & \cdots & \alpha_{n}\cdot m_{nJ}
\end{pmatrix}
\]

\end{document}
